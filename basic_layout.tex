\documentclass[12pt,a4paper]{article} %set font size to 12pt and use din A4 paper

%function to create global vars, which can be defined by 
%\<VARNAME>{<VALUE>} and accessed by \get<VARNAME>
\makeatletter
\newcommand*\DefVar[1]{\@namedef{#1}##1{\global\@namedef{get#1}{##1}}}
\makeatother

%create needed globvars
\DefVar{Title}
\DefVar{AuthorName}
\DefVar{AuthorMatNr}
\DefVar{AuthorMail}
\DefVar{AuthorMailLink}

%set default values
\Title{Your Document Title}
\AuthorName{Your Name}
\AuthorMatNr{}
\AuthorMail{Your eMail address}


%setup some needed vars
\AuthorMailLink{\href{mailto:\getAuthorMail}{\getAuthorMail}}

%include needed packages, set language and document geometry(padding)
\usepackage[utf8]{inputenc}
\usepackage[ngerman]{babel} %use new german dictionary to warp words
\usepackage{amsmath}
\usepackage{amsfonts}
\usepackage{amssymb}
\usepackage{graphicx}
\usepackage{lmodern}
\usepackage{fancyhdr}
\usepackage{lastpage}
\usepackage{hyperref}
\usepackage[usenames,dvipsnames,svgnames,table,colortbl]{xcolor}
\usepackage{colortbl} %colored tables
\usepackage{tabularx} %better tables
\usepackage{siunitx} %si units
\usepackage[german]{fancyref}
\usepackage[left=2.5cm,right=2.5cm,top=2.5cm,bottom=2.5cm]{geometry}

%setup siunit for german documents and geophysics
\sisetup{
	quotient-mode = fraction,
	fraction-function = \tfrac,
	output-decimal-marker = {,},
	per-mode = symbol,
	group-minimum-digits = 4,
	group-separator = \text{~},
	exponent-product = \cdot,
	output-product = \cdot
}
\DeclareSIUnit\gon{^{gon}}

%set line spacing
\linespread{1.50}

%don't draw ugly boxes around hyperlinks, just display them like normal text
\hypersetup{unicode=true,colorlinks=true,linkcolor=black,filecolor=black,urlcolor=black}

%setup page style
\pagestyle{fancy}
\fancyhead{} \fancyfoot{}

\fancyhf[HL]{
	\getAuthorName \\
	\footnotesize{\href{mailto:\getAuthorMail}{\getAuthorMail}}
}
\fancyhf[HC]{
	{\footnotesize Matr.Nr.:} $\getAuthorMatNr$
}
\fancyhf[HR]{
	\textsc{\getTitle} \\
	\footnotesize{}
}

\fancyhf[FR]{
	\thepage /\pageref{LastPage}
}

\author{
	\getAuthorName\\
	Matrikelnummer: \textit{\getAuthorMatNr}\\
	Email: \textit{\href{mailto:\getAuthorMail}{\getAuthorMail}}\\\\
}
\title{\getTitle}