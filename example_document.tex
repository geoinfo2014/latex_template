\documentclass[12pt,a4paper]{article} %set font size to 12pt and use din A4 paper

%function to create global vars, which can be defined by 
%\<VARNAME>{<VALUE>} and accessed by \get<VARNAME>
\makeatletter
\newcommand*\DefVar[1]{\@namedef{#1}##1{\global\@namedef{get#1}{##1}}}
\makeatother

%create needed globvars
\DefVar{Title}
\DefVar{AuthorName}
\DefVar{AuthorMatNr}
\DefVar{AuthorMail}
\DefVar{AuthorMailLink}

%set default values
\Title{Your Document Title}
\AuthorName{Your Name}
\AuthorMatNr{}
\AuthorMail{Your eMail address}

%setup some needed vars
\AuthorMailLink{\href{mailto:\getAuthorMail}{\getAuthorMail}}

%include needed packages, set language and document geometry(padding)
\usepackage[utf8]{inputenc}
\usepackage[T1]{fontenc}
\usepackage[ngerman]{babel} %use new german dictionary to warp words
\usepackage{amsmath}
\usepackage{amsfonts}
\usepackage{amssymb}
\usepackage{graphicx}
\usepackage{lmodern}
\usepackage{fancyhdr}
\usepackage{lastpage}
\usepackage{hyperref}
\usepackage[usenames,dvipsnames,svgnames,table,colortbl]{xcolor}
\usepackage{colortbl} %colored tables
\usepackage{tabularx} %better tables
\usepackage{siunitx} %si units
\usepackage{subfigure}
\usepackage{float}
\usepackage[german]{fancyref}
\usepackage[left=2.5cm,right=2.5cm,top=2.5cm,bottom=2.5cm]{geometry}

%setup siunit for german documents and geophysics
\sisetup{
	quotient-mode = fraction,
	fraction-function = \tfrac,
	output-decimal-marker = {,},
	per-mode = symbol,
	group-minimum-digits = 6,
	group-separator = \text{~},
	exponent-product = \cdot,
	output-product = \cdot
}
\DeclareSIUnit\gon{gon}
\DeclareSIUnit\mgon{\milli\gon}
\DeclareSIUnit\px{px}

%set line spacing
\linespread{1.50}

%smooth gray
\definecolor{smoothgray}{HTML}{EEEFFF}

%don't draw ugly boxes around hyperlinks, just display them like normal text
\hypersetup{unicode=true,colorlinks=true,linkcolor=black,filecolor=black,urlcolor=black}

%setup page style
\pagestyle{fancy}
\fancyhead{} \fancyfoot{}

\fancyhf[HL]{
	\getAuthorName \\
	\footnotesize{\href{mailto:\getAuthorMail}{\getAuthorMail}}
}
\fancyhf[HR]{
	\textsc{\getTitle} \\
	\footnotesize{}
}

\fancyhf[FL]{
	{\footnotesize Matr.Nr.:} $\getAuthorMatNr$
}
\fancyhf[FR]{
	\thepage /\pageref{LastPage}
}

\author{
	\getAuthorName\\
	Matrikelnummer: \textit{\getAuthorMatNr}\\
	Email: \textit{\href{mailto:\getAuthorMail}{\getAuthorMail}}\\\\
}
\title{\getTitle}

\AuthorName{Markus Mr.}
\AuthorMatNr{12345678}
\AuthorMail{github-contact@mr-pi.de}

\usepackage{listings} %source code package
\definecolor{CBackG}{HTML}{EEEFFF}
%\usepackage{caption}
%\DeclareCaptionFont{white}{\color{white}}
%\parbox[position][height][inner-pos]{width}{text}
%\DeclareCaptionFormat{listing}{\colorbox{black}{\parbox{\textwidth}{#1#2#3}}}
%\captionsetup[lstlisting]{format=listing,labelfont=white,textfont=white}
\lstset{
	numbers=left,
	backgroundcolor=\color{CBackG},
	frame=lines,
	basicstyle=\ttfamily\scriptsize,
	numberstyle=\tiny,
	keywordstyle=\bfseries\color{blue},
	stringstyle=\color{red},
	commentstyle=\color{DarkGreen},
	breaklines=true
}
\lstset{literate=%
	{Ö}{{\"O}}1
	{Ä}{{\"A}}1
	{Ü}{{\"U}}1
	{ß}{{\ss}}1
	{ü}{{\"u}}1
	{ä}{{\"a}}1
	{ö}{{\"o}}1
	{~}{{\textasciitilde}}1,
}

\lstset{language=TeX}

\Title{Geodäsie Template Uni Stuttgart}
\fancyhf[HC]{}
\author{
	\getAuthorName\\
	Email: \textit{\href{mailto:\getAuthorMail}{\getAuthorMail}}\\\\
}


\begin{document}
\maketitle %create fancy title
\tableofcontents %create table of content (sections)

\newpage %create new page for sections

\section{Usage}
\subsection{Anpassen des Templates}
Bevor Ihr das Template nutzen könnt, müsst Ihr es noch für euch Anpassen, hierzu müsst Ihr die Datei \texttt{basic\_data.tex} bearbeiten. Dort tragt Ihr an entsprechender Stelle euren Namen, eure Matrikelnummer und E-Mail adresse in die geschweiften Klammern ein. Dieses kann dann z.B. so aussehen:
\begin{lstlisting}
 \AuthorName{Max Musterman}
 \AuthorMatNr{12345678}
 \AuthorMail{muster@example.com}
\end{lstlisting}

\subsection{Einbinden der Templates}
Zum nutzen des Templates müsst Ihr schließlich noch die beiden Dateien \texttt{basic\_layout.txt} und \texttt{basic\_data.tex} einbinden.
Dieses könnt ihr am einfachsten machen, indem Ihr am Anfang eures Dokumentes folgendes schreibt:
\begin{lstlisting}
 \documentclass[12pt,a4paper]{article} %set font size to 12pt and use din A4 paper

%function to create global vars, which can be defined by 
%\<VARNAME>{<VALUE>} and accessed by \get<VARNAME>
\makeatletter
\newcommand*\DefVar[1]{\@namedef{#1}##1{\global\@namedef{get#1}{##1}}}
\makeatother

%create needed globvars
\DefVar{Title}
\DefVar{AuthorName}
\DefVar{AuthorMatNr}
\DefVar{AuthorMail}
\DefVar{AuthorMailLink}

%set default values
\Title{Your Document Title}
\AuthorName{Your Name}
\AuthorMatNr{}
\AuthorMail{Your eMail address}

%setup some needed vars
\AuthorMailLink{\href{mailto:\getAuthorMail}{\getAuthorMail}}

%include needed packages, set language and document geometry(padding)
\usepackage[utf8]{inputenc}
\usepackage[T1]{fontenc}
\usepackage[ngerman]{babel} %use new german dictionary to warp words
\usepackage{amsmath}
\usepackage{amsfonts}
\usepackage{amssymb}
\usepackage{graphicx}
\usepackage{lmodern}
\usepackage{fancyhdr}
\usepackage{lastpage}
\usepackage{hyperref}
\usepackage[usenames,dvipsnames,svgnames,table,colortbl]{xcolor}
\usepackage{colortbl} %colored tables
\usepackage{tabularx} %better tables
\usepackage{siunitx} %si units
\usepackage{subfigure}
\usepackage{float}
\usepackage[german]{fancyref}
\usepackage[left=2.5cm,right=2.5cm,top=2.5cm,bottom=2.5cm]{geometry}

%setup siunit for german documents and geophysics
\sisetup{
	quotient-mode = fraction,
	fraction-function = \tfrac,
	output-decimal-marker = {,},
	per-mode = symbol,
	group-minimum-digits = 6,
	group-separator = \text{~},
	exponent-product = \cdot,
	output-product = \cdot
}
\DeclareSIUnit\gon{gon}
\DeclareSIUnit\mgon{\milli\gon}
\DeclareSIUnit\px{px}

%set line spacing
\linespread{1.50}

%smooth gray
\definecolor{smoothgray}{HTML}{EEEFFF}

%don't draw ugly boxes around hyperlinks, just display them like normal text
\hypersetup{unicode=true,colorlinks=true,linkcolor=black,filecolor=black,urlcolor=black}

%setup page style
\pagestyle{fancy}
\fancyhead{} \fancyfoot{}

\fancyhf[HL]{
	\getAuthorName \\
	\footnotesize{\href{mailto:\getAuthorMail}{\getAuthorMail}}
}
\fancyhf[HR]{
	\textsc{\getTitle} \\
	\footnotesize{}
}

\fancyhf[FL]{
	{\footnotesize Matr.Nr.:} $\getAuthorMatNr$
}
\fancyhf[FR]{
	\thepage /\pageref{LastPage}
}

\author{
	\getAuthorName\\
	Matrikelnummer: \textit{\getAuthorMatNr}\\
	Email: \textit{\href{mailto:\getAuthorMail}{\getAuthorMail}}\\\\
}
\title{\getTitle}

 \AuthorName{Markus Mr.}
\AuthorMatNr{12345678}
\AuthorMail{github-contact@mr-pi.de}

\end{lstlisting}

Nun müsst ihr nur noch den Title eures Dokumentes festlegen, dieses geschieht durch den Befehl \texttt{\char`\\Title\{<Euer Titel>\}}, wobei \texttt{<Euer Titel>} durch den Dokumenten Titel ersetzt werden muss.

\subsection{Erstellen eines minimalistischen Dokumentes mit Deckblatt und Inhaltsverzeichnis}
Ein minimales Dokument welches dieses Template nutzt, muss die Layout Datei sowie Nutzer-Informationen einbinden(\textit{Zeile 1-2}) den Dokumenten Titel festlegen(\textit{Zeile 4}) und wie jedes Standard {\LaTeX} Dokument den eigentlichen Inhalt einleiten(\textit{Zeile 7}). Je nach bedarf kann dann noch der Title mit allen Informationen von euch(\textit{Titel, Name, E-Mail, Matrikelnummer und aktuellen Datum}) angelegt werden und ein Inhaltsverzeichnis erzeugt werden. Das ganze sieht dann in etwa so aus:
\pagebreak
\begin{lstlisting}
 \documentclass[12pt,a4paper]{article} %set font size to 12pt and use din A4 paper

%function to create global vars, which can be defined by 
%\<VARNAME>{<VALUE>} and accessed by \get<VARNAME>
\makeatletter
\newcommand*\DefVar[1]{\@namedef{#1}##1{\global\@namedef{get#1}{##1}}}
\makeatother

%create needed globvars
\DefVar{Title}
\DefVar{AuthorName}
\DefVar{AuthorMatNr}
\DefVar{AuthorMail}
\DefVar{AuthorMailLink}

%set default values
\Title{Your Document Title}
\AuthorName{Your Name}
\AuthorMatNr{}
\AuthorMail{Your eMail address}

%setup some needed vars
\AuthorMailLink{\href{mailto:\getAuthorMail}{\getAuthorMail}}

%include needed packages, set language and document geometry(padding)
\usepackage[utf8]{inputenc}
\usepackage[T1]{fontenc}
\usepackage[ngerman]{babel} %use new german dictionary to warp words
\usepackage{amsmath}
\usepackage{amsfonts}
\usepackage{amssymb}
\usepackage{graphicx}
\usepackage{lmodern}
\usepackage{fancyhdr}
\usepackage{lastpage}
\usepackage{hyperref}
\usepackage[usenames,dvipsnames,svgnames,table,colortbl]{xcolor}
\usepackage{colortbl} %colored tables
\usepackage{tabularx} %better tables
\usepackage{siunitx} %si units
\usepackage{subfigure}
\usepackage{float}
\usepackage[german]{fancyref}
\usepackage[left=2.5cm,right=2.5cm,top=2.5cm,bottom=2.5cm]{geometry}

%setup siunit for german documents and geophysics
\sisetup{
	quotient-mode = fraction,
	fraction-function = \tfrac,
	output-decimal-marker = {,},
	per-mode = symbol,
	group-minimum-digits = 6,
	group-separator = \text{~},
	exponent-product = \cdot,
	output-product = \cdot
}
\DeclareSIUnit\gon{gon}
\DeclareSIUnit\mgon{\milli\gon}
\DeclareSIUnit\px{px}

%set line spacing
\linespread{1.50}

%smooth gray
\definecolor{smoothgray}{HTML}{EEEFFF}

%don't draw ugly boxes around hyperlinks, just display them like normal text
\hypersetup{unicode=true,colorlinks=true,linkcolor=black,filecolor=black,urlcolor=black}

%setup page style
\pagestyle{fancy}
\fancyhead{} \fancyfoot{}

\fancyhf[HL]{
	\getAuthorName \\
	\footnotesize{\href{mailto:\getAuthorMail}{\getAuthorMail}}
}
\fancyhf[HR]{
	\textsc{\getTitle} \\
	\footnotesize{}
}

\fancyhf[FL]{
	{\footnotesize Matr.Nr.:} $\getAuthorMatNr$
}
\fancyhf[FR]{
	\thepage /\pageref{LastPage}
}

\author{
	\getAuthorName\\
	Matrikelnummer: \textit{\getAuthorMatNr}\\
	Email: \textit{\href{mailto:\getAuthorMail}{\getAuthorMail}}\\\\
}
\title{\getTitle}

 \AuthorName{Markus Mr.}
\AuthorMatNr{12345678}
\AuthorMail{github-contact@mr-pi.de}


 \Title{Muster Dokument}


 \begin{document}
 \maketitle %erzeugt das Deckblatt(den Titel mit allen wichtigen Informationen von euch)
 \tableofcontents %erzeugt das Inhaltsverzeichnis

 \newpage %springt auf eine neue Seite (kann bei kleinen Dokumenten auch wegfallen)

 \section{Usage} %erstellt einen Absatz
 %hier folgt der Inhalt des ersten Absatzes.
 
 \end{document} %beendet das Dokument
\end{lstlisting}


\section{Grundlagen}
\subsection{Mathematische Funktionen}
\subsubsection{Darstellung im Test}
Um in {\LaTeX} Funktionen im Text darzustellen, müsst ihr die Formel nur mit zwei \texttt{\$} Symbolen umschließen, das sieht dann z.B. so aus $x=100\si{\meter} \cdot 2\si{\meter}$.

\subsubsection{Darstellung größerer Funktionen (Zentriert)}
Längere Funktionen sollten nicht im Fließtext dargestellt werden sondern zentriert unter dem Text stehen. Um solche Funktionen darzustellen gibt es den \texttt{equation} block.
\subsubsection*{Beispiel}
\begin{lstlisting}
 Unter diesem Text steht eine Mathematische Formel
 \begin{equation}
     x=\frac{
         \sqrt[5]{600 \cdot }
     }{
         \int_5^{20} 5y
     }
     \cdot \left(\begin{array}{c}5\\6\end{array}\right)
 \end{equation}
\end{lstlisting}
Unter diesem Text steht eine Mathematische Formel
\begin{equation} \label{eq:einfache}
	x=\frac{
	    \sqrt[5]{600 \cdot }
	}{
		\int_5^{20} 5y
	}
	\cdot \left(\begin{array}{c}5\\6\end{array}\right)
\end{equation}

\subsubsection{Darstellung von Gleichungssystemen}
Zum Darstelen von Gleichungssystem wie z.B.
\begin{align}
\measuredangle \text{P3P2A}
	&= \arccos \left(
		\frac{\overline{\text{P3P2}}^2 + \overline{\text{P2A}}^2 - \overline{\text{P3A}}^2}
		     {2 \cdot \overline{\text{P3P2}} \cdot \overline{\text{P2A}}}
		\right) \\
	&= \arccos \left(
		\frac{6,934^2\si{\meter} + 7,002^2\si{\meter} - 9,911^2\si{\meter}}
		     {2 \cdot 6,934\si{\meter} \cdot 7,002\si{\meter}}
		\right) \\
	&= 1,5823 \\
	&= 1,5823 \cdot \frac{200}{\pi} = 100,7340\si{\gon}
\end{align}
gibt es den \texttt{align} block, bei den ihr dann mit Hilfe des \texttt{\&}-Symbols angeben könnt, anhand welchen Zeichens in einer Zeile die Formel ausgerichtet wird. Dabei richtet {\LaTeX} die Zeile immer anhand des auf den \texttt{\&} folgenden Symbols aus.
\subsubsection*{Beispiel}
\begin{lstlisting}
 \measuredangle \text{P3P2A}
 	&= \arccos \left(
 		\frac{\overline{\text{P3P2}}^2 + \overline{\text{P2A}}^2
 		    - \overline{\text{P3A}}^2}
 		     {2 \cdot \overline{\text{P3P2}} \cdot \overline{\text{P2A}}}
 		\right) \\
 	&= \arccos \left(
 		\frac{6,934^2\si{\meter} + 7,002^2\si{\meter} - 9,911^2\si{\meter}}
 		     {2 \cdot 6,934\si{\meter} \cdot 7,002\si{\meter}}
 		\right) \\
 	&= 1,5823 \\
 	&= 1,5823 \cdot \frac{200}{\pi} = 100,7340\si{\gon}
 \end{align}
\end{lstlisting}

\subsubsection{Formeln nicht durchnummerieren}
Um die Formel nicht durchnummerieren, genügt es den \texttt{equation} bzw. den \texttt{align} block einfach ein Asterisk anzustellen, so wird z.B. aus \fref{eq:einfache} das hier:
\begin{equation*}
	x=\frac{
	    \sqrt[5]{600 \cdot }
	}{
		\int_5^{20} 5y
	}
	\cdot \left(\begin{array}{c}5\\6\end{array}\right)
\end{equation*}
\begin{lstlisting}
 \begin{equation*}
 	x=\frac{
 	    \sqrt[5]{600 \cdot }
 	}{
 		\int_5^{20} 5y
 	}
 	\cdot \left(\begin{array}{c}5\\6\end{array}\right)
 \end{equation*}
\end{lstlisting}

\subsection{Darstellung von Einheiten}
Damit die Einheiten richtig dargestellt werden, ist das Paket \texttt{siunitx} geladen und schon auf Deutsche Darstellung umgestellt. Das wohl wichtigste ist, dass der Dezimalpunkt automatisch in Kommata umgewandelt werden (\textit{Vorausgesetzt, das der }\texttt{\char`\\num\{\}} oder \texttt{\char`\\SI\{\}\{\}}\textit{ Befehl genutzt wird}). Wer mehr über die Vorzüge dieses Pakets wissen möchte möge sich bitte einfach an mich wenden oder in die Doku schauen: \url{ftp://ftp.dante.de/tex-archive/macros/latex/exptl/siunitx/siunitx.pdf}. Dennoch hier die wichtigsten Funktionen.
\subsubsection{Zahlen}
Wenn ihr im Text Zahlen schreiben wollt nutzt das \texttt{\char`\\num\{\}} Kommando, so wird z.B:\\
\texttt{\char`\\num\{12345.6789e3\}} als \num{12345.6789e3} dargestellt oder \texttt{\char`\\num\{13 x 1/3\}} als \num{13 x 1/3}.

\subsubsection{Winkel}
Wenn ihr Winkel in Grad mit Sekunden und Minuten darstellen wollt, könnt hier das \texttt{\char`\\ang\{\}} Kommando nutzen, das ganze sieht dann wie Folgt aus:

\begin{tabularx}{\textwidth}{XX}
\rowcolor{CBackG}
 \texttt{\char`\\ang\{10\}} & \ang{10} \\
 \texttt{\char`\\ang\{10;12;\}} & \ang{10;12;} \\
\rowcolor{CBackG}
 \texttt{\char`\\ang\{10;12;20.4\}} & \ang{10;12;20.4}
\end{tabularx}\\
\\
Im weiteren ist auch die Einheit gon von mir vordefiniert wurden und kann wie folgt verwendet werden:

\begin{minipage}{0.5\textwidth}
\begin{lstlisting}[numbers=none]
$\SI{103.63}{\gon}$ oder $200\si{\gon}$
\end{lstlisting}
\end{minipage}
\begin{minipage}{0.5\textwidth}
$\SI{103.63}{\gon}$ oder $200\si{\gon}$
\end{minipage}


\end{document}
